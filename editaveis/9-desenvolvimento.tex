\chapter{Desenvolvimento da Ferramenta de Feedback Formativo - FeedFor}

\section{Metodologia de Desenvolvimento}

Para o desenvolvimento da ferramenta FeedFor, utilizou-se a metodologia Kanban, uma abordagem ágil que se mostrou eficaz para o gerenciamento do projeto. A escolha do Kanban visou garantir uma visão clara das atividades em andamento e permitir uma gestão eficiente das tarefas e prazos. A plataforma Jira foi empregada como ferramenta de apoio para a implementação do Kanban. No Jira, as atividades foram organizadas em quadros Kanban, facilitando a visualização do fluxo de trabalho e o controle do progresso das tarefas.

\section{Riscos e Impactos}

Durante o desenvolvimento do projeto FeedFor, diversos riscos foram considerados, alguns não previstos aconteceram e impactaram o andamento do trabalho. Entre os principais acontecimentos enfrentados, destacam-se:

\begin{itemize}
    \item A saída inesperada de um membro da equipe teve um impacto significativo na carga de trabalho e na divisão de tarefas. A quantidade de trabalho planejada para duas pessoas precisou ser redistribuída, o que resultou em uma reavaliação dos prazos e no aumento da carga de trabalho para o membro restante.
    \item A greve que afetou a Universidade de Brasília durante todo o semestre de 2024.1 causou interrupções significativas nas atividades acadêmicas e afetou a continuidade do projeto. A greve resultou em atrasos e na necessidade de ajustar o cronograma original.
    \item Um acidente sofrido pelo orientador resultou em uma licença médica durante o período da greve e algumas semanas após.
\end{itemize}

Os impactos desses acontecimentos incluíram a necessidade de reorganizar as tarefas e ajustar os prazos de entrega, além de enfrentar desafios adicionais na gestão do projeto. A falta de orientação e a alteração na divisão do trabalho contribuíram para a complexidade do gerenciamento das atividades, exigindo um esforço adicional para garantir a conclusão bem-sucedida da ferramenta FeedFor.

