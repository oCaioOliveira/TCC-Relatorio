\chapter[Introdução]{Introdução}

\section{Contexto}

No âmbito educacional da Universidade de Brasília, questionários aplicados nas disciplinas requisitos de software e métodos de desenvolvimento de software serão administrados através da plataforma Microsoft Forms. Serão aplicados regularmente ao longo do semestre e mensuram o desempenho dos discentes por meio de resultados quantitativos, ou seja, escalas numéricas, além de apresentar a quantidade de respostas certas e erradas visando avaliar o conhecimento dos alunos. 

\section{Problema}

Frequentemente os estudantes enfrentam dificuldades ao receberem resultados de avaliações que utilizam apenas classificações numéricas, pois essas designações, por si só, não fornecem uma compreensão clara do desempenho ou das áreas a serem aperfeiçoadas. Essas mensurações, embora forneçam uma noção geral, não oferecem informações específicas sobre os erros cometidos ou os tópicos que o estudante precisa revisar e consolidar. Isso pode resultar em frustrações e desafios para o estudante, por não entender claramente o resultado de uma avaliação. 

Portanto, o resultado das avaliações individuais apresentadas nas disciplinas requisitos de software e métodos de desenvolvimento de software é apenas quantitativo, uma solução limitada para os estudantes, uma vez que dá ênfase a um \textit{feedback} voltado para avaliações positivas ou negativas do que para um formato formativo. O qual poderia contribuir mais significativamente para a formação do discente e seu desenvolvimento dentro das matérias especificadas.

\section{Justificativa}

O presente trabalho beneficiará diretamente os estudantes ao oferecer \textit{feedbacks} formativos, os quais proporcionam uma análise aprofundada e construtiva do desempenho individual, orientando de maneira mais eficaz para o aprimoramento contínuo. Além disso, os professores e a instituição de ensino também se beneficiarão, podendo aprimorar suas práticas pedagógicas com base nos pontos levantados nos \textit{feedbacks}.

A literatura educacional enfatiza a importância de avaliações detalhadas e direcionadas para um \textit{feedback} formativo, visando promover uma aprendizagem significativa. As discussões no artigo "The Power of Feedback" de Hattie e Timperley \cite{hattie2007}, ressaltam a importância crucial do \textit{feedback} como um elemento fundamental no processo educacional ao apontar seu impacto significativo no desempenho dos alunos, conforme evidenciado por Hattie \cite{hattie2007}. Esse autor realizou uma síntese abrangente de mais de 500 meta-análises, envolvendo um impressionante contingente de 20 a 30 milhões de estudantes, elucidando o impacto substancial que o \textit{feedback} exerce sobre o progresso acadêmico. 

\section{Objetivos}

O objetivo geral do trabalho é desenvolver uma Inteligência Artificial capaz de aprimorar o processo de avaliação acadêmica, no que diz respeito as avaliações individuais das disciplinas, requisitos de software e métodos de desenvolvimento de software, utilizando inteligência artificial para construir \textit{feedbacks} formativos individualizados com base no desempenho dos estudantes.

Os objetivos específicos são:

\begin{enumerate}
  \item Identificar as necessidades existentes no atual processo avaliativo individual de estudantes nas disciplinas de requisitos de software e métodos de desenvolvimento de software, em relação aos \textit{feedbacks} recebidos. (Fase 1)
  \item Avaliar quais são as necessidades mais relevantes para os discentes. (Fase 1)
  \item Desenvolver uma ferramenta utilizando inteligência artificial que seja capaz de analisar o desempenho dos discentes e gerar \textit{feedbacks} formativos personalizados para cada aluno conforme as necessidades levantadas. (Fase 2)
\end{enumerate}

\section{Metodologias}

\begin{table}[!ht]
    \centering
    \begin{tabularx}{\textwidth}{|c|X|X|}
    \hline
        \textbf{Fase do Trabalho} & \textbf{Objetivos Específicos} & \textbf{Métodos e Técnicas} \\ \hline
        \multirow{2}{*}{Fase 1} 
        & Identificar as necessidades existentes no atual processo avaliativo individual de estudantes nas disciplinas de requisitos de software e métodos de desenvolvimento de software, em relação aos \textit{feedbacks} recebidos. 
        & Grupo Focal e Survey \\ \cline{2-3}
        & Avaliar quais são as necessidades mais relevantes para os discentes. 
        & Snowballing e Alfa de Cronbach \\ \hline
        Fase 2 
        & Desenvolver uma ferramenta utilizando inteligência artificial que seja capaz de analisar o desempenho dos discentes e gerar \textit{feedbacks} formativos personalizados para cada aluno conforme as necessidades levantadas. 
        & Kanban \\ \hline
    \end{tabularx}
    \caption{Tabela de objetivos e técnicas referentes a Fase 1 e 2 do projeto}
\end{table}

O estudo combina elementos qualitativos e quantitativos para alcançar uma compreensão abrangente do processo avaliativo acadêmico na Universidade de Brasília. A coleta de dados será realizada por meio de análise de resultados em formulários e entrevistas em grupos focais. Os participantes serão estudantes das disciplinas de requisitos de \textit{software} e métodos de desenvolvimento de \textit{software}, que serão selecionados com base em critérios predefinidos para garantir uma representação diversificada e relevante para os objetivos da pesquisa.

As entrevistas em grupos focais com os discentes, permitirão uma compreensão mais aprofundada das percepções dos estudantes, acerca dos desafios enfrentados, ao receberem resultados rasos sobre os questionários realizados. Além disso, serão uma fonte valiosa de \textit{insights} sobre suas sugestões para aperfeiçoar os feedbacks recebidos.

Essa abordagem mista, combinando pesquisa \textit{survey} e entrevistas em grupos focais de discentes nas disciplinas de requisitos de software e métodos de desenvolvimento de software, será essencial para obter uma visão holística das dificuldades enfrentadas pelos estudantes. Os dados coletados fornecerão uma base sólida para aprimorar o processo de avaliação, possibilitando a implementação da Inteligência Artificial, para oferecer feedbacks mais personalizados e eficazes.

\section{Estrutura do Trabalho}

Este trabalho está organizado em uma estrutura que visa oferecer uma análise detalhada e progressiva sobre o tema proposto. A seguir, são apresentados os principais capítulos que compõem esta pesquisa:

\begin{itemize}
    \item \textbf{Capítulo 1 - Introdução}: Apresenta uma visão geral do tema, os objetivos da pesquisa, a justificativa e a metodologia adotada.
    
    \item \textbf{Capítulo 2 - Referencial Teórico}: Constrói as bases teóricas do estudo, abrangendo desde conceitos fundamentais até modelos e teorias relacionadas ao tema específico.
    
    \item \textbf{Capítulo 3 - Metodologias}: Detalha o design da pesquisa, métodos e submétodos escolhidos, explicando a coleta e análise de dados.
    
    \item \textbf{Capítulo 4 - Execução da Pesquisa e Resultados da Fase 1}: Apresenta o processo de execução da pesquisa, incluindo cronograma e etapas do estudo, além da exposição dos resultados brutos.
    
    \item \textbf{Capítulo 5 - Análise dos Resultados da Fase 1}: Interpreta os resultados brutos, transformando-os em percepções e discutindo-os em relação aos objetivos e hipóteses do estudo.
    
    \item \textbf{Capítulo 6 - Análise de Viabilidade Técnica}: Enfoca a viabilidade prática da aplicação baseada em inteligência artificial para promover \textit{feedbacks} formativos, abordando objetivos técnicos, requisitos da ferramenta e tecnologias idealizadas.
    
    \item \textbf{Capítulo 7 - Execução e Resultados da Fase 2}: Apresenta um aglomerado de informações sobre a execução e o desenvolvimento do projeto na fase 2.

    \item \textbf{Capítulo 8 - Análise de Resultados da Fase 2}: Apresenta a análise e avaliação geral dos resultados alcançados pelo desenvolvimento do projeto na fase 2.

    \item \textbf{Capítulo 9 - Considerações Finais}: Apresenta informações sobre o desenvolvimento do projeto, apresentando as considerações finais sobre tudo que foi descoberto e desenvolvido.
\end{itemize}

