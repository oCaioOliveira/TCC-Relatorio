\chapter{Análise de Viabilidade Técnica}

\section{Introdução}

Neste capítulo, discutiremos a viabilidade técnica do desenvolvimento de uma Inteligência Artificial para aprimorar a avaliação individual nas disciplinas de requisitos de software e métodos de desenvolvimento de software na Universidade de Brasília. Exploraremos as tecnologias e ferramentas necessárias para analisar o desempenho dos alunos, e fornecer \textit{feedbacks} formativos.

\section{Objetivos Técnicos}

Os objetivos técnicos deste trabalho refletem em analisar o desempenho dos alunos em avaliações individuais das disciplinas requisitos de software e métodos de desenvolvimento de software, para gerar \textit{feedbacks} formativos por meio de uma inteligência artificial.

Existindo uma comunicação externa à plataforma utilizada nas disciplinas, o Microsoft Forms, sendo executada por meio de uma automação com a ferramenta Microsoft Power Automate, que provê a comunicação entre os dados das avaliações individuais dos alunos e os feedbacks formativos gerados pela inteligência artificial.

\section{Requisitos da Ferramenta}

A implementação para alcançar o objetivo técnico desejado pode exigir o uso de diversas ferramentas de inteligência artificial e \textit{frameworks}. Algumas considerações incluem:

\begin{itemize}
  \item Processamento de Linguagem Natural (PLN): para que, após a compreensão e análise de desempenho, possamos gerar \textit{feedback} textual de forma coerente e compreensível.
  \item Integração com API Externa: considerando a possibilidade de enviar dados para uma API externa que receberá os resultados das avaliações individuais dos estudantes e retornará o \textit{feedback} gerado pela IA.
\end{itemize}

\section{Tecnologias Idealizadas}

Levando em consideração os requisitos da ferramenta a ser construída, foi planejada a utilização das seguintes tecnologias em cada ambiente:

\begin{itemize}
  \item PLN: biblioteca NLTK (\textit{Natural Language Toolkit}), SpaCy, modelos de linguagem pré-treinados como BERT e ferramentas da OpenAI, como o modelo GPT (\textit{Generative Pre-trained Transformer}).
  \item API Externa: utilizar o \textit{framework} FastAPI, para desenvolver uma API em Python que se comunique com o modelo de inteligência artificial.
\end{itemize}

Além disso, será desenvolvido um fluxo de trabalho que integra o Microsoft Forms e a API externa. Os dados das avaliações individuais realizadas pelos discentes serão enviados automaticamente para a API por meio do Microsoft Power Automate. Após o processamento das informações, o feedback formativo gerado será disponibilizado para os alunos em um arquivo no formato PDF, acessível diretamente através da plataforma de e-mail.