\chapter{Conclusão}

\section{Lições Aprendidas}

O desenvolvimento do projeto FeedFor trouxe diversas lições valiosas, tanto no aspecto educacional quanto técnico. Uma das principais lições aprendidas é a importância do feedback pós-avaliação no processo de aprendizagem. Embora o ensino em sala de aula e as avaliações sejam cruciais, a continuidade do aprendizado após as avaliações é fundamental para o desenvolvimento do aluno. O objetivo do FeedFor é fornecer feedback formativo detalhado para poder ser um poderoso recurso para o aprimoramento contínuo dos estudantes.

Além disso, o projeto envolveu a exploração e o uso de ferramentas e bibliotecas novas, como a OpenAI API, que representaram um desafio técnico significativo. Dissecar a documentação da OpenAI API e integrar suas funcionalidades ao projeto foi uma experiência enriquecedora. A importância de compreender profundamente a documentação e ajustar os parâmetros de acordo com as necessidades específicas do projeto. Outras tecnologias, como a criação de PDFs e o envio de e-mails, também foram componentes novos que agregaram ao conhecimento técnico adquirido durante o desenvolvimento.

\section{Trabalhos Futuros}

Para o futuro do projeto FeedFor, existem várias oportunidades de evolução e expansão. Uma das direções para o futuro é a implementação e utilização da ferramenta em disciplinas. Ao aplicar o FeedFor em diferentes contextos acadêmicos, será possível coletar métricas e feedbacks dos estudantes para avaliar e ajustar o direcionamento dos estudos de forma mais eficaz.

Além disso, a integração com outros recursos educacionais e a adaptação da ferramenta para diferentes modalidades de ensino podem ser exploradas. O aprimoramento contínuo dos algoritmos de feedback e a inclusão de novas funcionalidades baseadas em feedbacks dos usuários são essenciais para manter a relevância e a eficácia da ferramenta.

\section{Conclusão do Projeto}

O projeto FeedFor alcançou um resultado satisfatório ao criar uma ferramenta robusta para a geração de feedbacks formativos. Com a capacidade de adaptação a diferentes contextos e funcionalidades, a ferramenta demonstra flexibilidade e potencial para ser aplicada em diversas áreas acadêmicas. A integração de tecnologias avançadas e a abordagem cuidadosa no design do sistema garantiram uma solução eficiente e escalável.

Embora tenha havido desafios ao longo do desenvolvimento, como a escolha da arquitetura e a integração com a OpenAI API, o desenvolvimento conseguiu superar essas dificuldades e entregar um produto que atende às necessidades iniciais e pode ser expandido para futuras aplicações. O FeedFor representa um avanço significativo no suporte ao aprendizado contínuo dos alunos e abre portas para futuras inovações no campo da educação.
