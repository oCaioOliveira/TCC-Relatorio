\chapter{Considerações Finais da Pesquisa}

\section{Resumo das Descobertas}

Este estudo buscou aprimorar o processo de avaliação acadêmica na Universidade de Brasília, com foco nas disciplinas de requisitos de software e métodos de desenvolvimento de software. As principais descobertas incluem a necessidade de feedbacks formativos detalhados e personalizados, em vez de simples avaliações numéricas. Utilizando ferramentas de Inteligência Artificial (IA) para a análise de dados, o estudo revelou que os alunos preferem feedbacks que oferecem explicações detalhadas, indicações de material de estudo e visualização clara dos gabaritos. A confiabilidade desses achados foi validada pelo coeficiente alfa de Cronbach, que obteve uma classificação alta. Adicionalmente, a análise da viabilidade técnica confirmou a possibilidade de implementar a solução de IA proposta, usando tecnologias como PLN e integração com APIs externas.

\section{Implicações}

As implicações destas descobertas são significativas para o campo da educação em ciências da computação. Ao enfatizar a importância de feedbacks formativos ricos em informações, este estudo sugere uma mudança de paradigma nas práticas avaliativas, movendo-se de uma abordagem quantitativa para uma mais qualitativa e enriquecedora. Essa mudança pode levar a um melhor engajamento dos alunos, facilitando uma compreensão mais profunda do conteúdo e incentivando a autorreflexão. Além disso, a implementação de soluções baseadas em IA na avaliação acadêmica levanta considerações importantes sobre a privacidade dos dados dos alunos e a necessidade de alinhamento ético nas práticas educacionais.

\section{Trabalhos futuros}

No desenvolvimento subsequente, especificamente para o Trabalho de Conclusão de Curso 2, a atenção será voltada para a implementação técnica de uma solução de Inteligência Artificial que ofereça feedbacks formativos detalhados e personalizados, mais especificamente, é proposto, para o TCC 2, que os formatos de feedbacks a serem trabalhados sejam "Exibir a explicação correspondente ao gabarito de cada questão" e "Visualizar os temas dos conteúdos que tiveram baixa taxa de acerto por meio de indicações textuais". Como foi descrito no capítulo 6, este próximo passo envolverá a construção e integração de um sistema de IA, com a plataforma do Moodle, que analise as avaliações dos alunos com eficiência e gere feedbacks formativos, mergulhando mais profundamente em tecnologias de Processamento de Linguagem Natural (PLN) e utilizando modelos avançados de linguagem.

A fase seguinte abrangerá testes abrangentes do sistema para validar sua eficácia. Esses testes irão verificar a precisão do feedback gerado, a usabilidade da interface e a integração com a plataforma de aprendizado existente. Paralelamente, será conduzido um estudo piloto em contextos reais de aprendizado, utilizando a solução de IA para coletar dados sobre a receptividade dos alunos, a eficácia do feedback e o impacto na aprendizagem e no engajamento dos estudantes.

Após a coleta desses dados, uma análise detalhada será realizada para entender a eficácia do sistema e identificar áreas para melhorias. Ajustes no sistema de IA poderão ser feitos com base no feedback dos usuários e na análise de desempenho. Além disso, será dada atenção especial à personalização e customização dos feedbacks, a fim de atender às necessidades individuais dos alunos, considerando diferentes estilos de aprendizagem e níveis de compreensão.

Também será essencial explorar as implicações éticas e de privacidade relacionadas ao uso de IA na educação, garantindo que a solução seja não apenas eficiente, mas também segura e ética. Por fim, a possibilidade de expandir a aplicação da solução de IA para outras disciplinas e contextos educacionais será avaliada, explorando sua adaptabilidade e eficácia em diferentes áreas de estudo.

Portanto, o TCC2 visa não apenas desenvolver uma solução tecnológica para feedbacks formativos, mas também contribuir significativamente para o campo da educação tecnológica, melhorando a experiência de aprendizado dos alunos e fornecendo insights valiosos para futuras inovações educacionais.

\section{Conclusão}

Em conclusão, este estudo ressalta a importância do feedback formativo no contexto educacional, demonstrando que a inclusão de feedback detalhado e personalizado, apoiado por tecnologias de IA, pode significativamente enriquecer a experiência de aprendizagem dos alunos. As descobertas deste trabalho servem como um ponto de partida para pesquisas futuras e para o desenvolvimento de práticas avaliativas mais eficazes e inclusivas na educação superior.