\chapter[Metodologias]{Metodologias}

\begin{table}[!ht]
    \centering
    \begin{tabularx}{\textwidth}{|X|X|X|X|}
    \hline
    \textbf{Fase do Trabalho} & \textbf{Objetivos Específicos} & \textbf{Métodos e Técnicas} & \textbf{Resultados} \\ \hline
    \multirow{2}{*}{Fase 1} & Identificar as necessidades existentes no atual processo avaliativo individual de estudantes nas disciplinas de requisitos de software e métodos de desenvolvimento de software, em relação aos \textit{feedbacks} recebidos & Grupo Focal e Survey & Lista de melhorias e feedbacks, além de obter dados quantitativos para realizar análises \\ \cline{2-4}
    & Avaliar quais são as necessidades mais relevantes para os discentes & Snowballing e Alfa de Cronbach & Obtenção de materiais relevantes para o trabalho e estimativa de confiabilidade do questionário \\ \hline
    Fase 2 & Desenvolver uma ferramenta utilizando inteligência artificial que seja capaz de analisar o desempenho dos discentes e gerar \textit{feedbacks} formativos personalizados para cada aluno conforme as necessidades levantadas & Kanban & Melhor organização e gerenciamento do trabalho e entregas ao longo do processo de desenvolvimento \\ \hline
    \end{tabularx}
    \caption{Tabela de objetivos, técnicas e resultados referentes a Fase 1 e 2 do projeto}
\end{table}

\section{Kanban}
\subsection{Introdução ao Kanban}
Originário do Japão, o Kanban foi inicialmente uma metodologia associada à Toyota na produção industrial, visando melhorar a eficiência, qualidade e fluxo de trabalho. Hoje, sua aplicação se estende a áreas diversas, como a gestão de projetos acadêmicos, por exemplo, na produção de TCCs \cite{corona2013review}.

O sistema Kanban se baseia em quadros visuais onde tarefas são representadas por cartões. Estes, por sua vez, transitam por etapas (ou colunas) no quadro, tornando transparente o progresso, identificação de gargalos e desafios. Segundo Erika Corona, o kanban utiliza a taxa de demanda para controlar a taxa de produção \cite{corona2013review}.

\subsection{Etapas do Kanban}

Para Corona\cite{corona2013review}, o Kanban, normalmente, segue alguns passos:

1. Mapear o fluxo, encontrando as atividades.

2. Expressar os requisitos através de um conjunto de características.

3. A depender das atividades e da equipe, deve-se estabelecer um limite máximo para finalizar as versões em andamento.

4. Configurar o quadro Kanban.

5. Elaborar a política para atribuir as atividades e tarefas e também para lidar com questões relacionadas ao fluxo.

6. Decidir o formato e a programação das reuniões.

7. Planejar o lançamento das versões do projeto.

Neste trabalho, o Kanban será utilizado para gerenciar e controlar as atividades a serem feitas e entregues, do início ao fim. A elaboração do quadro deve ser a primeira etapa do trabalho.

\section{Coleta de Dados}

\subsection{Survey}

Segundo Pinsonneault \cite{pinsonneault1993survey}, a abordagem de pesquisa por levantamento pode ser definida como o processo de coleta de dados ou informações sobre as características, comportamentos ou opiniões de um grupo específico de pessoas designado para representar uma população-alvo. Isso é geralmente realizado por meio de um instrumento de pesquisa, frequentemente na forma de um questionário.

Para Pinsonneault \cite{pinsonneault1993survey} a survey é apropriada quando:

1. As questões centrais de interesse sobre os fenômenos são “o que está acontecendo?”, e “como e por que isso está acontecendo?".

2. O controle das variáveis independentes e dependentes não é possível ou não é desejável.

3. Os fenômenos de interesse devem ser estudados em seu ambiente natural.

4. Os fenômenos de interesse ocorrem no tempo atual ou no passado recente.

Neste trabalho, utilizaremos o Survey para elaborar um questionário a fim de realizar análises quantitativas e qualitativas com o propósito de entender as preferências, quanto ao formato de feedback, dos discentes que participarão do questionário.

Nesta etapa é esperado criar o protótipo do questionário, para, posteriormente, passar pela validação por parte do professor orientador.

\subsection{Validação com Grupo Focal}

Para Caplan \cite{caplan1990using}, os grupos focais são “pequenos grupos de pessoas reunidos para avaliar conceitos ou identificar problemas”.

O objetivo central do grupo focal é identificar percepções, sentimentos, atitudes e
idéias dos participantes a respeito de um determinado assunto, produto ou atividade. Seus
objetivos específicos variam de acordo com a abordagem de pesquisa. Em pesquisas
exploratórias, seu propósito é gerar novas idéias ou hipóteses e estimular o pensamento do
pesquisador, enquanto que, em pesquisas fenomenológicas ou de orientação, é aprender
como os participantes interpretam a realidade, seus conhecimentos e experiências.\cite{dias2000grupo}

A primeira etapa do grupo focal é o seu planejamento. Nessa etapa deve ser
definido o objetivo da pesquisa, isto é, o que se pretende e quais as metas específicas a
serem alcançadas \cite{dias2000grupo}.

Segundo Dias \cite{dias2000grupo}, "... o moderador é responsável pela elaboração do guia de entrevista, a condução da discussão, a análise e o relato de seus resultados. Em certos casos, atua inclusive no recrutamento dos participantes".

Ainda na fase de planejamento, é escolhido o local mais apropriado para a
realização da reunião. A fim de facilitar a interação entre os participantes, é recomendável
um ambiente agradável, tranqüilo, sem quaisquer objetos que possam desviar a atenção do
grupo ou interromper a discussão, como telefones, por exemplo. A localização das pessoas
na sala deve facilitar o contato visual entre todos. Para isso, é comum a disposição de
cadeiras em círculo ou em torno de uma grande mesa redonda \cite{dias2000grupo}.

Neste trabalho, o grupo focal será realizado após a validação do questionário pelo professor orientador e tem como objetivo aprimorar o questionário, por meio de insights e novas ideias de questões e de formatos de feedback, providos pelos participantes.

\subsection{Avaliação da confiabilidade}

Apresentado por Lee J. Cronbach em 1951, o coeficiente $\alpha$ de Cronbach (assim como é cientificamente conhecido) é uma das estimativas da confiabilidade de um questionário que tenha sido aplicado em uma pesquisa, dado que todos os itens de um questionário utilizam a mesma escala de medição \cite{freitas2005avaliaccao}.

A fórmula para encontrar o $\alpha$ foi definida por Lee J. Cronbach:

\begin{equation}
\frac{k}{k-1} \left(1 - \frac{\sum_{i=1}^{k} \sigma_{i}^2}{\sigma_{T}^2}\right)
\end{equation}

Freitas e Rodrigues \cite{freitas2005avaliaccao}, sugerem a classificação da confiabilidade do coeficiente alfa de Cronbach de acordo com os seguintes limites:

A. $\alpha$ $\leq$ 0,30 – Muito baixa

B. 0,30 < $\alpha$ $\leq$ 0,60 - Baixa

C. 0,60 < $\alpha$ $\leq$ 0,75 - Moderada

D. 0,75 < $\alpha$ $\leq$ 0,90 - Alta

E. $\alpha$ > 0,90 – Muito alta

Neste trabalho, após a obtenção das respostas do questionário, será realizado o cálculo do alfa de Cronbach com o propósito de medir a consistência interna dos itens do questionário.

\subsection{Análise de dados}

Os dados obtidos com a realização da survey devem ser analisados por meio de ferramental estatístico para a obtenção das informações desejadas, devendo-se, para tanto, considerar o tipo de análise estatística aplicável às variáveis em estudo. As variáveis podem ser qualitativas, que têm como resultado atributos ou qualidades, ou quantitativas, que têm como resultado números de determinada escala \cite{freitas2000metodo}.

Neste trabalho, a análise de dados será feita de forma quantitativa e qualitativa, utilizando gráficos com o propósito de realizar uma análise profunda dos dados coletados, correlacionando-os com os objetivos e hipóteses do trabalho. 

Através da análise de dados

\section{Pesquisa em Artigos}

\subsection{Snowballing}

Segundo Wohlin \cite{wohlin2014guidelines}, o método de Snowballing refere-se ao uso da lista de referências bibliográficas de um artigo ou das citações de um artigo para identificar documentos adicionais. O snowballing pode ser dividido em "backward snowballing" e "forward snowballing".

Nas pesquisas em bases de dados, o primeiro passo é identificar palavras-chave e formular strings de pesquisa. Ao aplicar uma abordagem de snowballing, o primeiro desafio é identificar um conjunto inicial de documentos a serem usados para o procedimento de snowballing \cite{wohlin2014guidelines}.

O backward snowballing consiste em utilizar a lista de referências para identificar novos artigos a serem incluídos \cite{wohlin2014guidelines}.

Para Wohlin \cite{wohlin2014guidelines}, as iterações no backward snowballing possuem os seguintes passos:

1. Olhar para o título na lista de referências.

2. Verificar o local da referência.

3. Olhar para o resumo do artigo referenciado.

4, Olhar para o artigo completo das referências

O forward snowballing envolve a identificação de novos artigos com base naqueles que citam o artigo em análise \cite{wohlin2014guidelines}.

Para Wohlin \cite{wohlin2014guidelines}, as iterações no forward snowballing possuem os seguintes passos:

1. Olhar para o título do artigo que cita.

2. Verificar o resumo do artigo que cita.

3. Observar o local da citação no artigo.

3. Ler o artigo completo citado.

Ainda de acordo com Wohlin \cite{wohlin2014guidelines}, caso nenhum novo artigo seja encontrado, então o processo de snowballing terminou.

Neste trabalho, a técnica de snowballing será utilizada com o propósito de encontrar materiais relevantes ao trabalho, principalmente referente as metodologias utilizadas.

