\begin{resumo}[Abstract]
 \begin{otherlanguage*}{english}
   This work revolves around the educational environment at the University of Brasília, specifically focusing on the disciplines of Software Requirements and Software Development Methods, where individual assessments are conducted using the Microsoft Forms platform.

    The aim of this study is to develop an Artificial Intelligence system capable of enhancing individual assessments in these disciplines, enabling students to comprehend their mistakes and understand how to evolve in future assessments through formative feedback.

    To achieve the outlined objectives, a mixed research methodology was implemented, incorporating quantitative and qualitative elements. Furthermore, Kanban and Scrum software development methodologies will be used for the development of AI.

    Finally, the result is expected to be an AI tool, capable of analyzing information from the responses of students' individual assessment questionnaires (carried out on the Microsoft Forms platform), and generating formative \textit{feedbacks}.
   \vspace{\onelineskip}
 
   \noindent 
   \textbf{Key-words}: \textit{Feedback}. qualitative. quantitative. Artificial Intelligence. Microsoft Forms. tests. 
 \end{otherlanguage*}
\end{resumo}
