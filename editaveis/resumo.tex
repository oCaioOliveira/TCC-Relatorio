\begin{resumo}

    O contexto desse trabalho se refere ao ambiente educacional da Universidade de Brasília, nas disciplinas requisitos de \textit{software} e métodos de desenvolvimento de \textit{software}, onde são realizadas avaliações individuais dessas matérias, utilizando a plataforma Microsoft Forms. 

    O objetivo desse trabalho é desenvolver uma Inteligência Artificial capaz de aprimorar as avaliações individuais das disciplinas, permitindo que os alunos compreendam seus erros e saibam como evoluir em futuras avaliações por meio de \textit{feedbacks} formativos.

    Para atingir os objetivos delineados, foi implementada uma metodologia de pesquisa mista, incorporando elementos quantitativos e qualitativos. Além disso, para o desenvolvimento da IA foi utilizada a metodologia de desenvolvimento de software Kanban.

    Por fim, como resultado, uma ferramenta de IA, capaz de analisar as informações das respostas dos questionários de avaliação individual dos estudantes (realizados na plataforma Microsoft Forms), e gerar \textit{feedbacks} formativos.

 \vspace{\onelineskip}
    
 \noindent
 \textbf{Palavras-chave}: \textit{Feedback}. qualitativos. quantitativos. Inteligência Artificial. Microsoft Forms. questionários. 
\end{resumo}
