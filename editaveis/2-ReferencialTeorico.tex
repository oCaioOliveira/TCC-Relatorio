\chapter{Referencial Teórico}

\section{Conceitos Fundamentais}

\subsection{Avaliação Acadêmica}

Segundo Aline Oliva em "Avaliação Educacional: O que é e Importância" \cite{oliva2023}, a avaliação acadêmica é um componente central no processo educacional, fornecendo informações sobre o desempenho dos estudantes e, consequentemente, reorientando as estratégias pedagógicas dos professores. Tradicionalmente, ela tem sido vista como uma forma de medir a aprendizagem e atribuir notas, mas abordagens modernas a veem como uma ferramenta poderosa para promover a aprendizagem significativa. O \textit{feedback} educacional, por sua vez, é um componente crucial da avaliação, proporcionando informações detalhadas sobre o desempenho dos alunos, orientando avanços e estimulando a autorreflexão. 

\subsection{Feedback Formativo}

Conforme o artigo "Um Estudo sobre o Feedback Formativo na Avaliação em Matemática e sua Conexão com a Atribuição de Notas" \cite{vaz2021}, o \textit{feedback} formativo assume um papel vital no panorama educacional, diferenciando-se da avaliação somativa ao direcionar-se para a regulação da aprendizagem dos alunos. Enquanto a avaliação somativa busca certificar e classificar os estudantes, o feedback formativo oferece orientação contínua para impulsionar o desenvolvimento acadêmico. Essa abordagem visa identificar lacunas de compreensão e oferecer sugestões específicas para aprimorar o desempenho dos alunos ao longo do processo de aprendizagem, indo além das notas e buscando a construção contínua do conhecimento e das habilidades dos estudantes.

\subsection{Avaliação Individual}

Nas disciplinas de requisitos de software e métodos de desenvolvimento de software, a avaliação individual dos alunos desempenha um papel fundamental no acompanhamento do progresso e na verificação da compreensão dos conteúdos lecionados. Essas avaliações são realizadas periodicamente ao longo do semestre para medir o conhecimento adquirido pelos estudantes nos tópicos apresentados em aula.

Essas avaliações tradicionalmente se baseiam em resultados numéricos, destacando as notas obtidas pelos alunos, bem como a quantidade de questões respondidas correta e incorretamente. Os formatos de perguntas variam entre múltipla escolha, verdadeiro ou falso, ou inserir sentenças nas lacunas corretas. Esse modelo de avaliação permite uma análise mais ampla das habilidades e conhecimentos adquiridos, além de oferecer uma visão objetiva do desempenho individual de cada estudante ao longo do curso.

\subsection{Inteligência Artificial}

Segundo o artigo "O que é IA? Saiba mais sobre inteligência artificial" \cite{oracleia2023}, a inteligência artificial (IA) engloba aplicações capazes de realizar tarefas complexas anteriormente executadas por interação humana, como comunicação online com clientes e jogos de xadrez. Embora seja comumente confundida com subcampos como \textit{machine learning} (ML) e \textit{deep learning}, a IA representa um termo mais amplo.

Empresas têm investido em equipes de ciência de dados para extrair valor de várias fontes de dados por meio da IA, visando automatizar tarefas complexas e melhorar a experiência do usuário. Desenvolvedores utilizam a IA para conectar-se com clientes, identificar padrões e resolver problemas, embora seja necessário conhecimento em matemática e algoritmos para seu uso. Projetos iniciais de IA, como a criação de aplicativos simples, podem ajudar a compreender seus conceitos básicos e potenciais.

A IA tornou-se um pilar da inovação, fornecendo compreensão detalhada de dados, automação de tarefas complexas e aprimoramento do desempenho das empresas \cite{oracleia2023}. Seu uso tem sido direcionado para diversas finalidades, desde segurança até análises preditivas. Apesar dos desafios computacionais e de expertise, as empresas estão priorizando sua implementação, reconhecendo sua capacidade de otimizar operações e impulsionar resultados comerciais.

\subsection{Processamento de Linguagem Natural}

O Processamento de Linguagem Natural (PLN) é uma vertente do \textit{machine learning} que confere aos computadores a habilidade de interpretar e compreender a linguagem humana \cite{amazonnlp2023}. Essa tecnologia permite a análise de grandes volumes de dados de voz e texto de múltiplos canais de comunicação, como e-mails, mensagens, redes sociais, áudio e vídeo. Com o PLN, é viável processar esses dados, identificar intenções ou sentimentos nas mensagens e fornecer respostas em tempo real.

O PLN desempenha um papel crucial na análise eficiente de dados textuais e falados, lidando com variações linguísticas, gírias e irregularidades gramaticais típicas das interações diárias \cite{amazonnlp2023}. Empresas utilizam-no para diversas tarefas automatizadas, desde processamento de documentos extensos e análise de \textit{feedback} de clientes até a implementação de chatbots para atendimento ao cliente automatizado.

A integração do PLN em aplicações voltadas para o cliente possibilita uma comunicação mais eficaz com os clientes \cite{amazonnlp2023}. Um exemplo é o uso de chatbots para analisar e responder automaticamente a questões comuns dos clientes, o que reduz custos, libera os agentes de tarefas redundantes e melhora a satisfação do cliente.

O PLN envolve técnicas de pré-processamento, treinamento de modelos e implantação para ser eficaz \cite{amazonnlp2023}. Começando pela coleta e preparação de dados, passando pelo pré-processamento e treinamento dos algoritmos, até sua implementação para inferência em situações reais.

As tarefas de PLN incluem marcação de parte do discurso, desambiguação do sentido das palavras, reconhecimento de voz, tradução automática, reconhecimento de entidades nomeadas e análise de sentimento \cite{amazonnlp2023}. Cada uma destas tarefas desempenha um papel específico na compreensão e análise do texto e da fala humana.

\subsection{Plataforma Microsoft}

O Microsoft Forms é uma ferramenta online que permite a criação de formulários, questionários e pesquisas. Esta plataforma se destaca pela facilidade de uso e integração com outros serviços da Microsoft, como o OneDrive, Excel e Microsoft Power Automate, permitindo uma coleta e análise de dados eficaz. Professores podem utilizar o Microsoft Forms para criar avaliações, questionários e pesquisas, facilitando a entrega e avaliação de trabalhos. Além disso, os dados coletados podem ser facilmente compartilhados e analisados para acompanhar o rendimento dos alunos ao longo do curso.

O Microsoft Power Automate, anteriormente conhecido como Microsoft Flow, é uma plataforma de automação que permite criar fluxos de trabalho automáticos entre aplicativos e serviços para sincronizar arquivos, coletar dados, receber notificações, enviar requisições e muito mais. Com a integração do Power Automate e o Excel, é possível automatizar a coleta, análise e compartilhamento de dados. Por exemplo, os professores podem configurar fluxos que transferem automaticamente as respostas dos formulários do Microsoft Forms para planilhas do Excel, onde os dados podem ser organizados, analisados e visualizados de maneira prática e eficiente.

\subsection{OpenAI API}

De acordo com a sua documentação \cite{openai2024}, OpenAI API é uma poderosa ferramenta que oferece acesso a modelos avançados de inteligência artificial da empresa OpenAI, permitindo a criação de aplicações que podem compreender e gerar texto de maneira sofisticada. Esta API pode ser utilizada para diversos fins educacionais, como a criação de assistentes virtuais para ajudar alunos com dúvidas, a geração automática de conteúdo educacional e a análise de dados textuais. Por exemplo, professores podem utilizar a OpenAI API para desenvolver chatbots que auxiliem os alunos em tempo real, respondendo perguntas e fornecendo explicações adicionais sobre o material do curso.

\section{Modelos e Teorias Relacionadas}

No âmbito desta pesquisa, destacamos a Teoria da Aprendizagem Significativa de David Ausubel \cite{teorias} e a Teoria da Avaliação Formativa de Paul Black e Dylan Wiliam \cite{domingos2006} como fundamentais para embasar nosso estudo relacionado a avaliação acadêmica e \textit{feedbacks} formativos.

A Teoria da Aprendizagem Significativa, proposta por David Ausubel \cite{teorias}, sugere que a aprendizagem é mais eficaz quando os novos conhecimentos se ancoram em conceitos já existentes na estrutura cognitiva do indivíduo, relacionando-se com \textit{feedbacks} formativos ao indicarem como o estudante pode se aperfeiçoar em uma área de estudos, que, por realizar uma avaliação, pressupõe que são conceitos previamente estabelecidos.

A Teoria da Avaliação Formativa, desenvolvida por Paul Black e Dylan Wiliam \cite{domingos2006}, ressalta a importância de \textit{feedbacks} formativos no processo educacional. Essa teoria enfatiza que a avaliação não deve ser apenas um meio de medir o desempenho, mas também uma ferramenta para orientar e consolidar a aprendizagem. 

Essas teorias são essenciais para o embasamento teórico deste estudo, são apresentados conceitos valiosos para aprimorar o processo de avaliação acadêmica, como a aprendizagem significativa segundo Ausubel \cite{teorias} e a avaliação formativa por Black e Wiliam \cite{domingos2006}. Sua aplicação e entendimento aprofundado serão fundamentais para o desenvolvimento de práticas mais eficazes de avaliação.

\section{Trabalhos Correlatos}

A literatura existente abrange uma gama significativa de estudos relacionados ao papel dos feedbacks no processo de construção da aprendizagem significativa e na interação positiva da inteligência artificial com a formação acadêmica dos estudantes. Ao investigar pesquisas publicadas nesta área, é evidente o reconhecimento do impacto fundamental dos \textit{feedbacks} na educação. Os estudos apresentados a seguir não apenas destacam a importância crucial desses mecanismos de retroalimentação no contexto educacional, mas também exploram como a aplicação da inteligência artificial tem contribuído para aprimorar o desenvolvimento e desempenho dos estudantes ao longo de sua jornada acadêmica.

O artigo "O \textit{feedback} como recurso para a motivação e
avaliação da aprendizagem na educação a
distância", de 2009 \cite{flores2009}, apresenta o recurso do \textit{feedback} como instrumento de motivação e avaliação da aprendizagem na educação à distância. O trabalho enfatiza o papel desse recurso como ferramenta capaz de estimular o estudante a construir e reconstruir sua aprendizagem, num exercício constante de reflexão. Assim sendo, nosso trabalho busca ampliar essa visão e inserir a Inteligência Artificial como ferramenta tecnológica capaz de facilitar esse processo, aliando os dois recursos no aperfeiçoamento do processo avaliativo.

Em "Inteligência Artificial e Feedback: novo desafio em avaliação", de 2023 \cite{villasboas2023}, é possível aprender com o relato de estudos sobre o uso da inteligência artificial para gerar \textit{feedbacks} dos estudante sobre a atuação dos docentes nos EUA. Os docentes receberiam \textit{feedbacks} dos estudantes sobre seu trabalho, visando aperfeiçoar suas práticas. O que o trabalho não mostra é o uso desse recurso para favorecer a aprendizagem dos estudantes, justamente o que a nossa pesquisa procura explorar.



Por fim, no trabalho "\textit{Artificial intelligence in constructing personalized and accurate feedback systems for students}" \cite{xu2023}, encontramos  o uso da inteligência artificial na elaboração de \textit{feedbacks} aos estudantes no contexto educacional e o impacto positivo sobre a aprendizagem. O estudo vem ao encontro deste trabalho no sentido de demonstrar que o uso de \textit{feedbacks} na avaliação pode favorecer a aprendizagem significativa e o aperfeiçoamento dos processos avaliativos.
