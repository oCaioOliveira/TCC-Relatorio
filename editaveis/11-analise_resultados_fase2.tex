\chapter{Análise de Resultados da Fase 2}

\section{Avaliação Geral da Ferramenta FeedFor}

Nesta seção, analisaremos a eficácia e o desempenho da ferramenta FeedFor com base nos resultados obtidos durante a fase de testes. A ferramenta demonstrou ser eficiente na geração de \textit{feedbacks} formativos, especialmente em disciplinas relacionadas a Métodos de Desenvolvimento de Software e Requisitos de Software, onde o contexto de aplicação foi mais amplamente explorado.

O aumento do escopo para abranger qualquer disciplina também é um ponto positivo, uma vez que a arquitetura flexível e a modelagem de banco de dados permitiram fácil adaptação a diferentes contextos acadêmicos.

\section{Desempenho e Eficiência das Tecnologias Utilizadas}

Durante a fase 2, foi possível observar que a escolha do framework Django proporcionou uma estrutura robusta e facilitou a implementação de novas funcionalidades, como autenticação e geração de relatórios. A integração com a API da OpenAI e o uso do Celery para gerenciamento de tarefas assíncronas mostraram-se essenciais para manter a escalabilidade e a responsividade do sistema.

A integração com a API da OpenAI foi fundamental para o sucesso do projeto. Durante os testes, o uso do modelo GPT-3.5-turbo foi suficiente para atender às necessidades de geração de \textit{feedbacks} formativos, e o custo associado ao uso dessa tecnologia foi considerado aceitável dentro dos parâmetros de teste.

O uso do Celery em conjunto com o Redis como broker foi essencial para a implementação de tarefas assíncronas, garantindo que o sistema permanecesse responsivo mesmo com um grande volume de requisições. A análise de desempenho mostrou que essa arquitetura suportou bem o fluxo de trabalho, evitando gargalos e garantindo que os \textit{feedbacks} fossem gerados e enviados em tempo hábil.

\section{Impacto das Funcionalidades Secundárias}

As funcionalidades adicionais, como a possibilidade de reenviar \textit{feedbacks} e gerar relatórios de desempenho, tem um impacto positivo na aceitação do sistema pelos usuários. 

A funcionalidade de reenvio de \textit{feedback} permite que os professores revisem e enviem novamente os \textit{feedbacks} formativos, o que se mostra útil em situações onde os estudantes precisarem de reforço adicional ou onde houver necessidade de correções.

A capacidade de gerar relatórios detalhados, contendo porcentagens de acertos, desempenho por questão, e outras métricas, permite uma análise mais aprofundada do progresso dos estudantes e das áreas que necessitavam de mais atenção.

\section{Análise do Banco de Dados e da Modelagem}

A evolução do banco de dados ao longo das três fases do projeto permitiu uma flexibilidade maior e uma melhor adaptação ao novo escopo do projeto. A inclusão de tabelas adicionais e a melhoria dos relacionamentos entre elas resultaram em uma estrutura mais eficiente e escalável.

A criação da tabela de Respostas na segunda fase trouxe uma otimização significativa ao sistema, reduzindo o número de requisições redundantes à API da OpenAI. Essa melhoria não apenas reduziu os custos operacionais, mas também aumentou a velocidade de resposta da ferramenta.

A terceira fase do projeto, que incluiu a expansão para suportar múltiplas disciplinas e professores, foi fundamental para a utilização do projeto em um contexto mais amplo. A modelagem do banco de dados permitiu que a ferramenta se adaptasse facilmente a diferentes disciplinas, mantendo a consistência e a integridade dos dados.

\section{Análise da Arquitetura do Sistema}

A mudança de abordagem da Arquitetura Limpa para uma Arquitetura em Camadas simplificada se mostrou eficaz dado o contexto e as necessidades do projeto. Embora a Arquitetura Limpa tenha suas vantagens, a Arquitetura em Camadas foi suficiente para manter uma boa separação de responsabilidades e facilitar a manutenção com a equipe reduzida.

A Arquitetura em Camadas proporcionou uma estrutura clara e bem definida, o que facilitou a manutenção e a evolução do sistema. Embora a Arquitetura Limpa pudesse oferecer maior independência entre camadas, a decisão de simplificar a arquitetura se mostrou adequada diante das limitações de recursos e tempo.

\section{Conclusão da Análise de Resultados}

A análise dos resultados da Fase 2 do projeto FeedFor indica que a ferramenta está em um estado avançado de desenvolvimento, com funcionalidades bem definidas e eficientes. As tecnologias utilizadas se mostraram adequadas para o escopo do projeto, e as melhorias implementadas na segunda e terceira fases trouxeram benefícios tangíveis em termos de desempenho, escalabilidade e custo. A ferramenta está pronta para ser expandida e utilizada em um contexto mais amplo, oferecendo uma solução eficaz para a geração de \textit{feedbacks} formativos em diversas disciplinas.
